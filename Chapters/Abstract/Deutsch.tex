Eine der größten medizinischen Durchbrüche war die Entwicklung der Computertomographie, welche ein dreidimensionales Bild des menschlichen Körpers liefert. Weitere Forschungen konzentrieren sich nicht nur auf die Bildgebung, sondern auch auf die Bildverarbeitung. In den letzten Jahren wurde die künstliche Intelligenz  für die Bildverarbeitung eingeführt, indem künstliche Neural Networks zur Klassifizierung, Segmentierung und Erkennung von Objekten wie z.B. Organen, in Bildern eingesetzt wurden. Im Jahre 2021 gelang \citetitle{Isensee.2021} \cite{Isensee.2021} mit der Einführung des nnU-Net Networks, das andere Algorithmen auf dem Gebiet der biomedizinischen Segmentierung übertraf, ein wichtiger Durchbruch. Andere Deep-Learning Networks könnten jedoch genauere Vorhersagen treffen. Für Convolutional Neural Networks gibt es zwei verschiedene Arten von Frameworks: (a) einstufige Algorithmen und (b) zweistufige Algorithmen. Zweistufige Algorithmen schneiden bei der Segmentierung mehrerer Objekte besser ab, zum Beispiel bei der Segmentierung beider Nieren des menschlichen Körpers. Deshalb hat \citetitle{He.op.2017}  \cite{He.op.2017} einen hohen Forschungswert \cite{Shu.2020}. In diesem Zusammenhang analysiert diese Arbeit die Leistung des \citetitle{He.op.2017} Networks und konzentriert sich auf die Beantwortung folgender Forschungsfragen:

\begin{enumerate}
    \item Ist es möglich ein Mask R-CNN für die automatische Segmentierung auf heterogene CT-Daten zu trainieren, um klinischen Anforderungen gerecht zu werden?
    \item Welchen Grad der Übereinstimmung kann mit der manuellen Experten-Segmentierung erreicht werden und wie hoch ist die Laufzeit?
\end{enumerate}

Zur Beantwortung der Fragen wurde ein Mask R-CNN auf öffentlich zugängigen CT - Bildern trainiert. Das System stellte sich unter Verwendung von zweidimensionalen Operationen als nicht geeignet dar. Jedoch zeigte das Experiment die Limitationen des Systems, auf dessen Basis das System deutlich verbessert werden kann. Die Ergebnisse zeigen wie in der Forschung weiter vorzugehen ist.