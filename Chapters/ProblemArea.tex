As stated above, there are multiple applications for medical segmentation \cite{Aljabri.2022} , e.g. :
\begin{itemize}
    \item Surgery \cite{Howe.1999} \cite{Aljabri.2022}: Organ segmentation is important for surgical planning. Not only in traditional surgeries, but also crucial for different robotic solutions, which can assist a surgery by improving precision, stability and dexterity. 
    
    \item Diseases Detection:  Using organ segmentation for anatomic analysis, it can be used to detect various conditions as vertebrae compressions, alzheimer, schizophrenia, heart disorder, glaucoma or retinal vein occlusions.
    
    \item Treatment Planning: Depending on the disease, organ segmentation can help with different types of diseases, e.g.  hepatic disease intervention or diabetes.
    
    \item Cancer detection: An essential application of organ segmentation is the use for cancer detection. It is the basis of automated analysis for different types of cancer, e.g. lung, brain or breast cancer.
    
    \item Radiotherapy: Segmentation is a required process for radiotherapy, a treatment method for cancer. Radiation is used to kill cancer cells but it can also harm nearby \ac{OAR}.  Hence, radiation should be applied accurately to reduce post-treatment complication and therefore requires precise \ac{OAR} segmentation.

\end{itemize}

A special form of radiotherapy is the radionuclide therapy also referred as \ac{MRT}. Unlike traditional radiotherapy, using external beams, radionuclide therapy uses radionuclides to be introduced into the body and to target the tumor from within. With the EURATOM directive 59/2013 \cite{.25Nov22} dosimetry  for treatments are required to be personalized in order to to avoid under dosage or over dosage \cite{DellaGala.2021}. Several treatment planning systems provide automated dosimetry calculations but require accurate segmentation, for reasons described above. However, manual contouring is highly time-consuming and labor intensive. Additionally, there are only a limited amount of experts qualified for doing manual segmentation. Because of these reasons, automated organ segmentation is of high interest. Some systems provide automated segmentation capabilities to some extent. E.g. QDOSE \cite{.23Aug22} from ABX-CRO advanced pharmaceutical services Forschungsgesellschaft mbH \cite{.25Nov22b} allows manual and automated segmentation for dosimetry calculation. An advantage of  QDOSE \cite{.23Aug22} is its possibility to correct existing segmentation under the responsibility of the of the final user. Many different algorithms are used for automated segmentation by different systems and vendors. The congruence between automated and manual expert segmentation differs by each algorithm and is therefor subject of research. Generally the quality needed to be useful for medical purposes are a challenge. Other challenges are the morphological variations, differences in location and intensity  values either between organ within a body or among different patients. Furthermore, there are also noise factors from various sources as patient movement or implants. Traditionally, atlas based and statistical shape based models were researched \cite{Kaur.2022}. Although they seemed promising, they proved to be insufficient \cite{Kaur.2022} and the focus shifted towards deep learning based methods. According to a survey \cite{Altini.2022} the drawback of deep learning methods are the need for a large quantity of training data. Medical data is also considered sensitive by the \ac{GDPR} and the usage as training data even more so a challenge. Nevertheless, in 2021 \citetitle{Isensee.2021} \cite{Isensee.2021} by \citeauthor{Isensee.2021} had an huge impact in the field of biomedical segmentation and is now considered the state of the art framework. In spite of nn-unet\cite{Isensee.2021} success, this thesis valuates if Mask R-CNN \cite{He.op.2017} could outperform the current standard. 